\documentclass{article}


% Language setting
% Replace `english' with e.g. `spanish' to change the document language
\usepackage[utf8]{inputenc}
\usepackage{amsmath}

% Set page size and margins
% Replace `letterpaper' with `a4paper' for UK/EU standard size
\usepackage[letterpaper,top=2cm,bottom=2cm,left=3cm,right=3cm,marginparwidth=1.75cm]{geometry}

% Useful packages
\usepackage{amsmath}
\usepackage{graphicx}
\usepackage[colorlinks=true, allcolors=blue]{hyperref}

\title{Complejidad Y Optimizaciòn Problema del Periódico}
\author{Nombres integrantes}
\date{Diciembre 01, 2022}



\begin{document}

\maketitle

\newpage
\tableofcontents
\newpage

\section{Modelo de la Instancia}
\subsection {Parámetros}
Sea 
\newline
$$ T = \{internacionales, nacionales, locales, deportes, cultura\} $$
\newline
El conjunto de los temas de los artículos del periódico tal que  $t \in T$.
\newline

Sea 
$$
M = \{5, 4, 2, 2, 1\}
$$
\newline
El conjunto del mínimo número de páginas de los temas.
\newline

Sea 
$$
N = \{9, 7, 5, 4, 3\}
$$
\newline
El conjunto del máximo número de páginas de cada tema.
\newline

Y sea 
$$
L = \{1500, 2000, 1000, 1500, 750\}
$$
\\
El conjunto de los lectores promedio por página.
\newline
\begin{itemize}
    \item $M_{t}$: representa el mínimo número de paǵinas del tema $t$.
    \item $N_{t}$: representa el máximo número de paǵinas del tema $t$.
    \item $L_{t}$: representa el promedio de lectores por página del tema $t$.
    \item $p$: representa el número total de páginas del periódico.
\end{itemize}

\subsection {Variables}
\begin{itemize}
    \item $Cantidad_{t}$: representa el número de paǵinas del tema $t$. $\forall t \in T, Cantidad_{t} \geq 0$.
\end{itemize}
\subsection {Restricciones}
\begin{itemize}
    \item La cantidad de páginas del tema $t$ debe satisfacer:

    $M_{t} \leq Cantidad_{t} \leq N_{t}$

    \item La cantidad de páginas totales debe satisfacer:

    $\sum_{t \in T} (Cantidad_{t}) \leq t$ 
\end{itemize}

\subsection {Función Objetivo}
Maximice: $\sum_{t \in T} Cantidad_t * L_t $

\section{Modelo Genérico}
\subsection {Parámetros}
\subsection {Variables}
\subsection {Restricciones}
\subsection {Función Objetivo}





\end{document}
